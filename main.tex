
%% This is file `sample-sigconf-authordraft.tex',
%% generated with the docstrip utility.
%%
%% The original source files were:
%%
%% samples.dtx  (with options: `all,proceedings,bibtex,authordraft')
%% 
%% IMPORTANT NOTICE:
%% 
%% For the copyright see the source file.
%% 
%% Any modified versions of this file must be renamed
%% with new filenames distinct from sample-sigconf-authordraft.tex.
%% 
%% For distribution of the original source see the terms
%% for copying and modification in the file samples.dtx.
%% 
%% This generated file may be distributed as long as the
%% original source files, as listed above, are part of the
%% same distribution. (The sources need not necessarily be
%% in the same archive or directory.)
%%
%%
%% Commands for TeXCount
%TC:macro \cite [option:text,text]
%TC:macro \citep [option:text,text]
%TC:macro \citet [option:text,text]
%TC:envir table 0 1
%TC:envir table* 0 1
%TC:envir tabular [ignore] word
%TC:envir displaymath 0 word
%TC:envir math 0 word
%TC:envir comment 0 0
%%
%% The first command in your LaTeX source must be the \documentclass
%% command.
%%
%% For submission and review of your manuscript please change the
%% command to \documentclass[manuscript, screen, review]{acmart}.
%%
%% When submitting camera ready or to TAPS, please change the command
%% to \documentclass[sigconf]{acmart} or whichever template is required
%% for your publication.
%%
%%


\documentclass[sigconf,authordraft]{acmart}

%%
%% \BibTeX command to typeset BibTeX logo in the docs
\AtBeginDocument{%
  \providecommand\BibTeX{{%
    Bib\TeX}}}

%% Rights management information.  This information is sent to you
%% when you complete the rights form.  These commands have SAMPLE
%% values in them; it is your responsibility as an author to replace
%% the commands and values with those provided to you when you
%% complete the rights form.
%%\setcopyright{acmlicensed}
%%\copyrightyear{2018}
%%\acmYear{2018}
%%\acmDOI{XXXXXXX.XXXXXXX}
%% These commands are for a PROCEEDINGS abstract or paper.
\acmConference[Conference acronym 'WISSA]{Wissenschaftliches Arbeiten}{February 04,
  2025}{Dresden, SN}
%%
%%  Uncomment \acmBooktitle if the title of the proceedings is different
%%  from ``Proceedings of ...''!
%%
%%\acmBooktitle{Woodstock '18: ACM Symposium on Neural Gaze Detection,
%%  June 03--05, 2018, Woodstock, NY}
%%\acmISBN{978-1-4503-XXXX-X/18/06}


%%
%% Submission ID.
%% Use this when submitting an article to a sponsored event. You'll
%% receive a unique submission ID from the organizers
%% of the event, and this ID should be used as the parameter to this command.
%%\acmSubmissionID{123-A56-BU3}

%%
%% For managing citations, it is recommended to use bibliography
%% files in BibTeX format.
%%
%% You can then either use BibTeX with the ACM-Reference-Format style,
%% or BibLaTeX with the acmnumeric or acmauthoryear sytles, that include
%% support for advanced citation of software artefact from the
%% biblatex-software package, also separately available on CTAN.
%%
%% Look at the sample-*-biblatex.tex files for templates showcasing
%% the biblatex styles.
%%

%%
%% The majority of ACM publications use numbered citations and
%% references.  The command \citestyle{authoryear} switches to the
%% "author year" style.
%%
%% If you are preparing content for an event
%% sponsored by ACM SIGGRAPH, you must use the "author year" style of
%% citations and references.
%% Uncommenting
%% the next command will enable that style.
%%\citestyle{acmauthoryear}
%%
%% end of the preamble, start of the body of the document source.

%% EDIT VON PAUL
\usepackage[ngerman]{babel}
\newcommand{\anf}[1]{\glqq{}#1\grqq{}}  % Anführungszeichen

\begin{document}



%%
%% The "title" command has an optional parameter,
%% allowing the author to define a "short title" to be used in page headers.
\title{Quantencomputer in der Kryptografie}

%%
%% The "author" command and its associated commands are used to define
%% the authors and their affiliations.
%% Of note is the shared affiliation of the first two authors, and the
%% "authornote" and "authornotemark" commands
%% used to denote shared contribution to the research.
\author{Daniel Joseph Heinrich Haus}
\email{s3005733@ba-sachsen.de}
\affiliation{
  \institution{DHSN}
  \city{Dresden}
  \state{Sachsen}
  \country{Deutschland}
}
\author{Ben Kaiser}
\email{s3005541@ba-sachsen.de}
\affiliation{
  \institution{DHSN}
  \city{Dresden}
  \state{Sachsen}
  \country{Deutschland}
}
\author{Paul Zenker}
\email{s3005664@ba-sachsen.de}
\affiliation{
  \institution{DHSN}
  \city{Dresden}
  \state{Sachsen}
  \country{Deutschland}
}
\author{Felix Brockwitz}
\email{s3005540@ba-sachsen.de}
\affiliation{
  \institution{DHSN}
  \city{Dresden}
  \state{Sachsen}
  \country{Deutschland}
}
\author{Matheo Zoeke}
\email{s3005785@ba-sachsen.de}
\affiliation{
  \institution{DHSN}
  \city{Dresden}
  \state{Sachsen}
  \country{Deutschland}
}

%%
%% By default, the full list of authors will be used in the page
%% headers. Often, this list is too long, and will overlap
%% other information printed in the page headers. This command allows
%% the author to define a more concise list
%% of authors' names for this purpose.
\renewcommand{\shortauthors}{DJHH, BK, PZ, FB, MZ}

%%
%% The abstract is a short summary of the work to be presented in the
%% article.
\begin{abstract}
  A clear and well-documented \LaTeX\ document is presented as an
  article formatted for publication by ACM in a conference proceedings
  or journal publication. Based on the ``acmart'' document class, this
  article presents and explains many of the common variations, as well
  as many of the formatting elements an author may use in the
  preparation of the documentation of their work.
\end{abstract}

%%
%% The code below is generated by the tool at http://dl.acm.org/ccs.cfm.
%% Please copy and paste the code instead of the example below.
%%
\begin{CCSXML}
  <ccs2012>
     <concept>
         <concept_id>10002978.10002979.10002984</concept_id>
         <concept_desc>Security and privacy~Information-theoretic techniques</concept_desc>
         <concept_significance>500</concept_significance>
         </concept>
   </ccs2012>
\end{CCSXML}
  
\ccsdesc[500]{Security and privacy~Information-theoretic techniques}

%%
%% Keywords. The author(s) should pick words that accurately describe
%% the work being presented. Separate the keywords with commas.
\keywords{Security, Cryptography, RSA, Shor´s Algorithm, AES, Quantum Computers}
%% A "teaser" image appears between the author and affiliation
%% information and the body of the document, and typically spans the
%% page.
\begin{teaserfigure}
  \includegraphics[width=\textwidth]{sampleteaser}
  \caption{ Google Quantum AI, 2019.}
  \Description{Berichtigter Quantencomputer: Neuartiger Chip korrigiert viele seiner Fehler}
  \label{fig:teaser}
\end{teaserfigure}


%%
%% This command processes the author and affiliation and title
%% information and builds the first part of the formatted document.
\maketitle

\section{Einleitung}
Hier sollte eine Einleitung stehen (Motivationen).

\section{Daniels Sektion - Quanten Computer Allgemein}
Ist assymetrische Verschlüsselung.\\


\section{Kryptografie\textendash Grundlagen}
Verschlüsselungsalgorithmen werden grundlegend in zwei Gruppen unterteilt.
Die erste Gruppe ist symmetrische Verschlüsselung, welche Grundlegend zum Verschlüsseln von Daten
oder Texten verwendet wird. Diese verwendet zum ver\textendash und entschlüssen den gleichen Schlüssel.
Im Gegensatz dazu steht die asymmetrische Verschlüsselung, welche zum ver\textendash und entschlüssen
zwei unterschiedliche Schlüssel verwendet. Diese werden Public Key (Verschlüsselung) und 
Private Key (Entschlüsselung) genannt. Als Beispiel dazu folgend der HTTPS\textendash Verbindungsaufbau
\begin{enumerate}
    \item Der Handshakeprozess: Dieser Prozess erfolgt mit asymmetrischer Verschlüsselung. 
    Am Anfang des Verbindungsaufbau fordert der Client
    das Zertifikat des Servers an. Folgend prüft der Client das Zertifikat über eine Zertifizierungsstelle.
    Folgend wird für die weitere Kommunikation ein Schlüssel vom Client generiert. Dieser wird
    mit dem öffentlichen Schlüssel des Servers verschlüsselt. Der Server wird diesen folgend mithilfe des 
    eigenen Private Keys entschlüsseln und für die folgende symmetrische Verschlüsselung nutzen. Damit
    wurde auf sichere Weise ein "Passwort" übertragen.
    \item Kommunikation: In diesem Schritt erfolgt das Senden von tatsächlichen Daten. Zum Beispiel den
    HTML Code der Seite. Zum ver\textendash und entschlüsseln dieser Daten wird der Schlüssel verwendet,
    welcher in Schritt 1 generiert wurde.
\end{enumerate}
\section{Pauls Sektion - RSA}
RSA ist ein asymmetrischer Verschlüsselungsalgorithmus. Die Sicherheit von RSA beruht
auf der Faktorisierung von Primzahlen, da dieses ein NP-Hard Problem ist\cite{moolchad_leveraging_nodate}.


 
\section{Infos}
\setlength{\parindent}{0pt}

Symmetrische Blockchiffre\\
Schlüssellängen von 128, 192 oder 256 Bit\\
Die zu verschlüsselnde Datengröße beträgt jeweils pro Block 128 Bit.\\

Anwendungen in VPNs, HTTPS und WLAN-Verschlüsselung (z. B. WPA2).\\
AES basiert auf mehreren Runden von Transformationen, die Daten verschlüsseln. Die Anzahl der Runden hängt von der Schlüssellänge ab (128-Bit-Schlüssel: 10 Runden).\\

Es gibt jeweils einen anderen Schlüssel pro Runde, welcher von dem Hauptschlüssel berechnet wird.\\

Erfunden vom National Institute of Standards and Technology (NIST) im Jahr 2001.\\

AES ist ein Block Cipher.\\
Ein Block Cipher nimmt einen festen Bit-Wert ein und gibt einen festen Bit-Wert aus.\\

AES-Blöcke sind in einem 4 x 4 Byte (128 Bit) Block angelegt:\\

\[
\begin{bmatrix}
c_0  & c_4  & c_8  & c_{12} \\
c_1  & c_5  & c_9  & c_{13} \\
c_2  & c_6  & c_{10} & c_{14} \\
c_3  & c_7  & c_{11} & c_{15}
\end{bmatrix}
\]

Jede Verschlüsselungsrunde besteht aus 4 Schritten:\\
\begin{itemize}
    \item SubBytes
    \item ShiftRows
    \item MixColumns
    \item Add Round Key
\end{itemize}

Die letzte Runde hat keinen MixColumns-Schritt.\\

\subsection{SubBytes}

\subsection{Shift Rows}

\noindent
\flushleft
\(
\begin{bmatrix}
c_0  & c_1  & c_2  & c_3  \\
c_4  & c_5  & c_6  & c_7  \\
c_8  & c_9  & c_{10} & c_{11} \\
c_{12} & c_{13} & c_{14} & c_{15}
\end{bmatrix}
\quad \rightarrow \quad
\begin{bmatrix}
c_0  & c_1  & c_2  & c_3  \\
c_5  & c_6  & c_7  & c_4  \\
c_{10} & c_{11} & c_8  & c_9  \\
c_{15} & c_{12} & c_{13} & c_{14}
\end{bmatrix}
\)
\flushleft
\subsection{MixColumns}
\subsection{Add Round Key}

\section{Shor’s Algorithmus: Funktionsweise und mathematische Grundlage}
\label{sec:shor}

Shor’s Algorithmus ist ein Quantenalgorithmus, der entwickelt wurde, um 
effizient die Primfaktoren einer großen Zahl $N$ zu bestimmen. Dieses 
Problem bildet die Grundlage vieler kryptografischer Verfahren wie RSA, 
da es für klassische Computer äußerst schwierig ist, große Zahlen in 
ihre Primfaktoren zu zerlegen. Shor’s Algorithmus hingegen nutzt die 
Eigenschaften von Quantencomputern, um diese Aufgabe exponentiell 
schneller zu lösen. \cite{shor}

\subsection{Ablauf des Algorithmus}

\begin{enumerate}
    \item \textbf{Auswahl einer Basis $a$:}  
    Zunächst wird eine Zahl $a$ gewählt, wobei $1 < a < N$ gilt. Diese 
    Basis dient als Ausgangspunkt für die nachfolgenden Berechnungen.

    \item \textbf{Periodenfindung:}  
    Der Algorithmus sucht die kleinste positive Periode $r$, sodass die 
    Kongruenzgleichung  
    \[
    a^r \mod N = 1
    \]  
    erfüllt ist. Die Periode $r$ ist der Schlüssel zur Faktorisierung von $N$.  
    \begin{itemize}
        \item Es wird sichergestellt, dass $r > 0$ ist, da eine Periode von 
        $r = 0$ keine Aussagekraft hat.
        \item Zur Bestimmung von $r$ könnte man klassisch eine Wertetabelle 
        für $r$ und die entsprechenden Werte $z = a^r \mod N$ erstellen. Da 
        diese Methode jedoch ineffizient ist, wird auf einem Quantencomputer 
        die \hyperref[sec:QFT]{Quanten-Fourier-Transformation (QFT)} verwendet, um $r$ effizient 
        zu berechnen.
    \end{itemize}

    \item \textbf{Verwendung der Periode $r$:}  
    Sobald $r$ bekannt ist, wird geprüft:  
    \begin{itemize}
        \item Falls $r$ ungerade ist, beginnt der Algorithmus erneut mit 
        einem anderen $a$.
        \item Ist $r$ gerade, werden die Primfaktoren von $N$ mit den 
        folgenden Formeln berechnet:  
        \[
        f_1 = \text{ggT}\left(a^{r/2} - 1, N\right) \quad \text{und} \quad f_2 = \text{ggT}\left(a^{r/2} + 1, N\right).
        \]  
        Der größte gemeinsame Teiler ($\text{ggT}$) wird dabei mithilfe 
        des Euklidischen Algorithmus ermittelt. \cite{shor_klassisch} \cite{shor_klassisch2} 
    \end{itemize}
\end{enumerate}



Der letzte Divisor, bevor der Rest $0$ wird, ist der gesuchte größte gemeinsame Teiler: \cite{euklid}
\[
\text{ggT}(132, 28) = 4. 
\] 

Im Hinblick auf unsere Forschungsfrage lässt sich festhalten, dass Shor's Algorithmus eine 
Methode darstellt, mit der die Berechnung eines kryptografischen Schlüssels effektiv rückgängig 
gemacht werden kann. Die Sicherheit vieler aktueller Verschlüsselungsverfahren, wie RSA, basiert 
auf der Schwierigkeit, eine große Zahl in ihre Primfaktoren zu zerlegen. Shor's Algorithmus macht 
es jedoch möglich, diese beiden Primfaktoren effizient zu bestimmen, sofern ein ausreichend 
leistungsfähiger Quantencomputer zur Verfügung steht. Dadurch wird das Fundament der zugrunde 
liegenden Verschlüsselung geschwächt, da die ursprüngliche Annahme, dass die Faktorisierung ein 
praktisch unlösbares Problem sei, durchbrochen wird.

\subsection{Fourier-Transformation: Zerlegung von Signalen in Frequenzen}
Die Fourier-Transformation ist ein Algorithmus, mit einer Komplexität von \(O\left(n^2\right) \), der häufig genutzt wird, um ein Signal in ein Spektrum von Frequenzen zu zerlegen, also in eine kontinuierliche Menge an Gewichten, sodass die kontinuierliche, gewichtete Summe aus Sinus und Kosinus Funktionen unterschiedlicher Frequenzen ist. Die am häufigsten genutzte Variante ist die diskrete Fourier-Transformation, die nicht kontinuierlich ist und somit nur endliche Frequenzen erkennt. \cite{don_h_johnson_58_2017}

\subsubsection{\textbf{Diskrete Fourier-Transformation: DFT}}\label{sec:DFT}~\newline
DFT operiert an einer diskreten Folge an komplexen Zahlen

\(a = (a_0,\dots,a_{N-1}) \in \mathbb{C}^N\) (\(a_k \in \mathbb{C}\)) und gibt eine Folge\\
\(\hat{a} = (\hat{a}_0,\dots,\hat{a}_{N-1}) \in \mathbb{C}^N\) (genannt \anf{diskret Fourier-Transformierte})
wobei \[\hat{a}_k = \sum_{j=0}^{N-1}e^{-2\pi \imath\cdot\frac{jk}{N}}\cdot a_j \quad \text{für}\quad k = 0,\dots,N-1\] oder als Matrix-Vektor-Produkt: \cite{wiki-discrete-fourier-transformation}
\[\hat{a} = W\cdot a \quad \text{mit} \quad W[k,j] = e^{-2\pi \imath \cdot\frac{jk}{N}}\]

\subsubsection{\textbf{Interpretation der Fourier-Transformation:}}
\begin{itemize}
	\item \textbf{Bedeutung des Ergebnisses:}\\
	Mit den diskret Fourier-Transformierten lässt sich eine Funktion konstruieren, die an den stellen \(n = 0,\dots,N-1\) die originalen Funktionswerte liefert.
	\[f[n] = \sum_{k=0}^{N-1}\left(\text{Re}(\hat{a}^\prime_k)\cos\left(\frac{2\pi kn}{N}\right)+\text{Im}(\hat{a}^\prime_k)\sin\left(\frac{2\pi kn}{N}\right)\right)\]
	wobei \(\hat{a}\) zur Vereinfachung und besseren Relation normiert wird mit \(\hat{a}^\prime = \frac{\hat{a}}{N}\).
	\(f[n]\) ist eine Summe von Kosinus- und Sinusfunktionen mit einer Frequenz von \(2\pi\cdot\frac{k}{N}\), wobei für jeden Wert \(\hat{a}^\prime_k\) der Realteil Koeffizient der jeweiligen Kosinusfunktion und der negative Imaginärteil Koeffizient der jeweiligen Sinusfunktion ist.
	Eine alternative Interpretation der Werte der diskret Fourier-Transformierten ist, das \(\left\lvert \hat{a}^\prime_k\right\rvert \) der Amplitude und die Phase \(\quad\arg(\hat{a}^\prime_k) = \arctan\left(\frac{\text{Im}(\hat{a}^\prime_k)}{\text{Re}(\hat{a}^\prime_k)}\right)\quad \) entspricht.
	\[f[n] = \sum_{k=0}^{N-1}\left\lvert\hat{a}^\prime_k\right\rvert\cdot\cos\left(\frac{2\pi kn}{N}+\arg(\hat{a}^\prime_k)\right)\]
	Dies ist eine Zerlegung der Eingangssequenz in Frequenzen.
	\item \textbf{Alternative Berechnung der diskreten Fourier Transformation:}\\
	Eine menschenfreundlichere Methode kann man erhalten, wenn man erhalten, indem man die Eulersche Formel anwendet:
	\[e^{-2\pi \imath \cdot\frac{jk}{N}} = \cos\left(\frac{2\pi jk}{N}\right) + \imath\cdot\sin\left(\frac{2\pi jk}{N}\right)\]
	Nach dieser Umstellung lässt sich folgender Algorithmus herleiten:
	\begin{enumerate}
		\item \textbf{Vergleichsfunktion definieren:}\\%Assymetrische Klammern in diesen Teil sind, weil sie Intervalle sind
		Der Algorithmus funktioniert, indem man die gegebene Sequenz mit Sinus- und Cosinuskurven von verschiedenen Frequenzen vergleicht. Da der Cosinus den Real- und der Sinus den Imaginärteil stellt und diese sich nicht gegenseitig beeinflussen, wenn \(a \in \mathbb{R}^N\), kann man diese auch getrennt beachten.
		\[f_k(x) = \cos\left(\frac{2\pi k}{N}\cdot x\right) + \imath \cdot\sin\left(\frac{2\pi k}{N}\cdot x\right) \]
		Die Winkelfunktionen sollen im Intervall \([0,N)\), also im Intervall der Eingangssequenz, \(k\) ganze Perioden durchlaufen. Der Cosinus (Sinus ist analog) durchläuft eine Periode im Intervall \([0,2\pi)\). Wenn man das Argument \(x\) mit \(2\pi\), multipliziert wird das Periodenintervall zu \([0,1)\). Durch Division mit \(N\) wird das Intervall zum verlangten \([0,N)\). Um \(k\) Perioden im Intervall zu erhalten, Multipliziert man mit \(k\), sodass das Intervall \(\left[0,\frac{N}{k}\right) ~k\)-mal in \([0,N)\) passt.
		\item \textbf{Vergleichssequenzen erstellen:}\\
		Mit der Funktion können wir die \(k\)-te Vergleichssequenz erhalten, indem wir \(f\) an denselben Stellen wie \(a\) abtasten.
		\[b_k = \left(f_k(0),f_k(1),\dots,f_k(N-1)\right) \]
		\item \textbf{Vergleichen:}\\
		Um zu erfahren, welchen Teil die \(k\)-te Frequenz in der Eingangssequenz spielt, nimmt man die Summe der Produkte von jedem Element \(a_j\) der Eingangssequenz mit den korrespondierenden \({(b_k)}_j\), d.h. das Skalarprodukt
		\[\hat{a}_k = a \cdot b_k\]
		\item \textbf{Spezialfall \(k = 0\):}\\
		Schon in der Ausgangsformel kann man sehen, das	für \(k = 0\) gilt: \[e^{-2\pi \imath \cdot\frac{j\cdot 0}{N}} = e^0 = 1\]
		weshalb sich \(\hat{a}_0\) zu
		\[\hat{a}_0 = \sum_{j=0}^{N-1} a_j\] zusammenfassen lässt. Womit \(\hat{a}^\prime_0 \) dem arithmetischen Mittel von \(a\) entspricht.
	\end{enumerate}
 	%Python algo
	%j aj âj
	%0 1  9 + 0j
	%1 2  0 + 0j
	%2 1  0 + 0j
	%3 2 -3 + 0j
	%4 1  0 + 0j
	%5 2  0 + 0j
	%import numpy as np
	%a = np.array([1,2,1,2,1,2])
	%ao = [0,0,0,0,0,0]
	%ao[0] = sum(a)#O(n)
	%for i in range(1,6):#O(n*z)
	%	#z = 3*O(n) = O(n)
	%	test = np.arange(0,2*np.pi*i,2*np.pi*i/6)#O(n)
	%	ao[i] = a.dot(np.cos(test))#O(n)
	%	ao[i] += 1j*a.dot(np.sin(test))#O(n)
	%#O(n²)
	%for n in ao:
	%	print(np.round(n,5))
\end{itemize}\cite{yt_reducible}
\subsubsection{\textbf{Quanten-Fourier-Transformation: QFT}}\label{sec:QFT}~\newline
QFT ist die Anwendung von \hyperref[sec:DFT]{DFT} in auf Quantencomputern. Sie hat \(n\) Qubits als Eingang, womit die Eingangssequenz aus \(N = 2^n\) Basiszuständen besteht: \(\left(|0\rangle,|1\rangle,\dots,|N-1\rangle\right)\). Der Quantenschaltkreis ist in Abbildung~\ref{fig:QFT_n_Qubits} zu sehen.
\begin{figure}[hb]
	\centering
	\includegraphics[width=0.45\textwidth]{sections/felix/Q_fourier_nqubits.png}
	\caption{QFT für n Qubits (ohne Umkehrung der Reihenfolge der Zustände der Ausgaben)\cite{wiki-q-fourier-nqubitspng}}
	
	\label{fig:QFT_n_Qubits}
\end{figure}
\\Wobei \(\left[0.x_1 x_2 \dots x_n\right]\) Notation für \(\frac{x_1}{2}+\frac{x_2}{4}+\cdots+\frac{x_n}{2^n}\) ist.
Verwendet werden das Hadamard-Gatter \(H = \frac{1}{\sqrt{2}}\begin{pmatrix}
	1 &  1\\
	1 & -1
\end{pmatrix}\) und Kontrollierte Phasengatter \(R_m = \begin{pmatrix}
	1 & 0\\
	0 & e^{\frac{2\pi\imath}{2^m}}\\
\end{pmatrix}\).
Eine Schaltung für \(3\) Qubits und somit \(2^3 = 8 = N\) ist in Abbildung~\ref{fig:QFT_3_Qubits} zu sehen.\cite{wiki-list-quantum-gates}
\begin{figure}[hb]
	\centering
	\includegraphics[width=0.45\textwidth]{sections/felix/Q_fourier_3qubits.png}
	\caption{QFT für 3 Qubits (ohne Umkehrung der Reihenfolge der Zustände der Ausgaben)\cite{wiki-q-fourier-3qubitspng}}
	\label{fig:QFT_3_Qubits}
\end{figure}
\\Diese Schaltung lässt sich mithilfe der Identitätsmatrix \(I = \begin{pmatrix}
	1 & 0 \\
	0 & 1 
\end{pmatrix}\) und der Tauschmatrix \(S = \begin{pmatrix}
	1 & 0 & 0 & 0\\
	0 & 0 & 1 & 0\\
	0 & 1 & 0 & 0\\
	0 & 0 & 0 & 1
\end{pmatrix}\) in der Gleichung
\begin{multline}
	W = (S\oplus I)\cdot (I \oplus H \oplus I) \cdot (R_2 \oplus I) \cdot (I \oplus S) \cdot (I\oplus R_3)\\ \cdot (S \oplus I) \cdot (I \oplus H \oplus I) \cdot (R_2 \oplus I) \cdot (H \oplus I \oplus I)
\end{multline}
\[QFT_8 |x\rangle = W \cdot \left(|x_1\rangle \oplus |x_2\rangle \oplus |x_3\rangle\right)\]
\[\text{oder auch als}\quad QFT_8 |x\rangle = \frac{1}{\sqrt{8}}\sum_{j=0}^{8-1} e^{\frac{2\pi \imath x j}{8}}|j\rangle\]
darstellen, sodass für ein allgemeines \(N = 2^n\) gilt:
\[QFT_N |x\rangle = \frac{1}{\sqrt{N}}\sum_{j=0}^{N-1} e^{\frac{2\pi \imath x j}{N}}|j\rangle\]
\cite{wiki-quanten-fouriertransformation}
Dies unterscheidet sich von der Formel für die \hyperref[sec:DFT]{DFT} nur
\begin{enumerate}
	\item in der Anwendung (auf Quantenzustände statt auf Vektoren/Sequenzen)
	\item den Vorzeichen des Exponenten und
	\item einen Normalisierungsfaktor (Die Linearkombination von Quantenzuständen muss immer den Betrag \(1\) haben).
\end{enumerate}
Trotz dieser Unterschiede enthält \(QFT_N |x\rangle\) Frequenzinformationen.\\
Wendet man QFT auf eine von \hyperref[sec:shor]{Shor's Algorithmus} generierten Sequenz an, erhält ein Basiszustand eine signifikant höhere Wahrscheinlichkeit, der der Periode \(r\) entspricht, sodass man nach einer Messung auf diese schließen kann.



\section{Fazit}
Die Analyse der Auswirkungen von Quantencomputern auf Verschlüsselungsalgorithmen
zeigt deutliche Bedrohungen auf die IT-Sicherheit. Insbesondere asymmetrische 
Verschlüsselungsverfahren wie RSA sind durch Shore's Algorithmus gefährdet. 
Der Grund liegt in der Fähigkeit von Quantencomputern, mit diesem Algorithmus
die Primfaktoren einer Zahl, wie zum Beispiel ein 2048 Bit RSA Schlüssel, 
effizient zu berechnen, wie in Sektion \ref{sec:shor} beschrieben wurde.\\
Auch AES ist durch den \anf{Grover's Algorithmus} bedroht, welcher die Rechenzeit 
halbiert, um den Schlüssel zu finden. Allerdings bietet dieser 
nicht wie Shore's Algorithmus eine exponentiell schnellere Berechnung und ist somit
grundlegend noch als sicher zu betrachten.\\
Bezüglich der ersten Forschungsfrage lässt sich festhalten, dass symmetrische 
Verschlüsselungsalgorithmen wie AES weniger stark betroffen sind als 
asymmetrische Verfahren. Während RSA durch Shor's Algorithmus praktisch 
gebrochen werden kann (siehe Sektion \ref{sec:shor}), bietet AES,
wie in Sektion \ref{sec:aes} dargestellt wurde, durch seine komplexe Struktur aus SubBytes, 
ShiftRows, MixColumns und der Rundenschlüssel auch gegenüber Quantencomputern 
noch Sicherheit, wenn auch mit erhöhten Anforderungen an die Schlüssellänge.\\
Die zweite Forschungsfrage hinsichtlich \anf{Store now, decrypt later-Angriffen}
zeigt die Dringlichkeit des Problems: Selbst wenn 
aktuelle Quantencomputer noch nicht leistungsfähig 
genug sind, sollten Unternehmen bereits möglichst zeitnah 
Maßnahmen ergreifen, da verschlüsselte Daten gespeichert 
und später entschlüsselt werden könnten, wie in Sektion 
\ref{sec:quantencomputer} dargestellt wurde.\\

\subsection{Ausblick}
Die Forschung im Bereich der Quantenkryptographie stellt eine erhebliche 
Herausforderung für die IT-Sicherheit dar, da RSA praktisch gebrochen werden 
kann. Auch andere weit verbreitete Alternative Algorithmen wie \anf{Elliptic Curve \textendash ED25519},
welche in diesem Paper nicht betrachtet wurden,
sind durch den Shor's Algorithmus gefährdet. Dies betrifft auch indirekt symmetrische
Verschlüsselungsalgorithmen, da diese oft in Kombination 
mit asymmetrischen Verfahren, wie in Abbildung \ref{fig:http_verbindungsaufbau} dargestellt, 
verwendet werden. Das ist vor allem für die Kommunikationstechnik von Relevanz, da nun
Verbindungen kompromittiert werden können, da symmetrische Schlüssel mithilfe
von asymmetrischen Verfahren ausgetauscht werden.\\




%%
%% The next two lines define the bibliography style to be used, and
%% the bibliography file.
\bibliographystyle{ACM-Reference-Format}
\bibliography{sample-base}


%%
%% If your work has an appendix, this is the place to put it.
\appendix

\end{document}
\endinput
%%
%% End of file `sample-sigconf-authordraft.tex'.
