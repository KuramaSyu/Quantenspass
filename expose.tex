\documentclass[12pt]{article} % Specifies the document class and font size

% Packages
\usepackage[utf8]{inputenc} % For UTF-8 character encoding
\usepackage[T1]{fontenc} % For better font rendering
\usepackage{geometry} % To customize page dimensions
\geometry{a4paper, margin=1in} % A4 paper with 1-inch margins
\usepackage{graphicx} % For including images
\usepackage{amsmath} % For math symbols and equations
\usepackage{hyperref} % For hyperlinks
\usepackage{setspace} % For adjusting line spacing

% Document information
\title{Basic LaTeX Template}
\author{Your Name}
\date{\today}

\begin{document}

% Title and author
\maketitle

% Section 1
\section{Introduction}
Die Entwicklung leistungsfähiger Quantencomputer stellt klassische Verschlüsselungsverfahren wie AES 
und RSA vor neue Herausforderungen. Insbesondere Algorithmen wie Shors Algorithmus könnten diese Methoden 
in absehbarer Zeit kompromittieren, da sie es ermöglichen, große Zahlen – eine zentrale Grundlage der 
Sicherheit klassischer Kryptosysteme – effizient zu faktorisieren. \\
Angesichts dieser Bedrohung 
wächst das Interesse an Quantenkryptografie, die auf den Gesetzen der Quantenmechanik basiert 
und Sicherheit unabhängig von der Rechenleistung eines Angreifers verspricht.

Ein zentraler Ansatz der Quantenkryptografie ist die Quanten-Schlüsselverteilung (Quantum Key Distribution, QKD), die eine abhörsichere Übertragung von Schlüsseln ermöglicht. Diese Technologien stellen einen wichtigen Schritt dar, um künftigen Bedrohungen durch Quantencomputer zu begegnen und die Sicherheit digitaler Systeme langfristig zu gewährleisten. In diesem Paper wird die Motivation für die Quantenkryptografie dargelegt und untersucht, wie sie als Lösung für die Schwächen klassischer Kryptografie dienen kann.

Forschungsfrage: 
Wie können Quantencomputer in der Kryptographie genutzt werden?
Welche Auswirkungen hat dies auf aktuelle Verschlüsslungsalgorithmen?

\section{Zielstellung}
Ziel dieser Arbeit ist es, die Funktionsweise von Quantencomputern zu erläutern und deren 
Einsatzmöglichkeiten in der Kryptographie zu untersuchen. Dabei sollen insbesondere die Auswirkungen 
auf aktuelle Verschlüsselungsalgorithmen wie RSA und AES betrachtet werden.

\section{Aktueller Forschungsstand}
Der Forschungsstand zu Quantencomputern ist noch neu. 
% Including an image
\begin{figure}[h!]
\centering
\includegraphics[width=0.5\textwidth]{example-image} % Replace with your image file
\caption{An example image.}
\label{fig:example}
\end{figure}

% Section 3
\section{Conclusion}
This is the conclusion section. Summarize your findings or key points here.

% References
\begin{thebibliography}{9}
\bibitem{sample} Author Name, \textit{Book Title}, Publisher, Year.
\end{thebibliography}

\end{document}
