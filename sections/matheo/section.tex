\section{Funktionsweise vom AES Algorithmus}
\setlength{\parindent}{0pt}

Advanced Encryption Standard (AES) ist ein symmetrischer Blockchiffre Algorithmus zur 
Verschlüsselung sensibler Daten wie VPN Verbindungen, HTTPS Anfragen 
(siehe Abbildung \ref{fig:http_verbindungsaufbau}) und WLAN (WPA2). 
Entwickelt wurde er 2001 vom NIST. AES arbeitet mit Schlüssellängen von 128, 192 oder 256 Bit, 
wobei längere Schlüssel sicherer sind. Die Blockgröße beträgt stets 128 Bit, da AES eine Blockchiffre 
ist. Die Anzahl der Runden hängt von der Schlüssellänge ab: 
\begin{itemize}
    \item 128 Bit: 10 Runden
    \item 192 Bit: 12 Runden
    \item 256 Bit: 14 Runden
\end{itemize}

Jeder 128-Bit-Block wird als 4x4-Matrix dargestellt:
\[
\begin{bmatrix}
c_0  & c_4  & c_8  & c_{12} \\
c_1  & c_5  & c_9  & c_{13} \\
c_2  & c_6  & c_{10} & c_{14} \\
c_3  & c_7  & c_{11} & c_{15}
\end{bmatrix}
\]

Jede Runde umfasst die Schritte \textit{SubBytes}, \textit{ShiftRows}, \textit{MixColumns} und \textit{Add Round Key}. In der letzten Runde wird \textit{MixColumns} übersprungen.\cite{AES_Algorithmus_2}\cite{AES_Algorithmus_3}\cite{AES_Algorithmus}\cite{Blockchiffre}

\subsection{S-Box}
Die S-Box (Substitution-Box) dient zur nichtlinearen Byte-Substitution, um die Beziehung zwischen Schlüssel und Chiffretext zu verschleiern. Jedes Eingangsbyte \(b\) (außer 0) wird im Galois-Feld \(GF(2^8)\) inversiert:
\[
b' = b^{-1} \mod (x^8 + x^4 + x^3 + x + 1)
\]
Dabei ist \(x^8 + x^4 + x^3 + x + 1\) das standardmäßige irreduzible Polynom. Das Ergebnis wird durch eine lineare Transformation angepasst:
\[
s_i = A \cdot b_i + c
\]
wobei \(A\) eine $8 \times 8$-Matrix ist und \(c\) eine Konstante.\cite{Endliche_körper}\cite{S_Box}

\subsection{SubBytes}
\textit{SubBytes} ersetzt jedes Byte des Datenblocks durch den entsprechenden Wert aus der S-Box, wobei die Bytewerte als Adressen dienen. Beispiel einer S-Box:
\begin{table}[H]
    \resizebox{0.3\textwidth}{!}{
    \begin{tabular}{|c|c|c|c|c|c|c|c|}
    \hline
    63 & 7C & 77 & 7B & F2 & 6B & 6F & C5 \\
    \hline
    30 & 01 & 67 & 2B & FE & D7 & AB & 76 \\
    \hline
    CA & 82 & C9 & 7D & FA & 59 & 47 & F0 \\
    \hline
    AD & D4 & A2 & AF & 9C & A8 & 51 & A3 \\
    \hline
    40 & 8F & 92 & 9D & 38 & F5 & BC & B6 \\
    \hline
    DA & 21 & 10 & FF & F3 & D2 & CD & 0C \\
    \hline
    13 & EC & 5F & 97 & 44 & 17 & C4 & A5 \\
    \hline
    33 & 85 & 45 & F9 & 02 & 7F & 50 & 3C \\
    \hline
    \end{tabular}
    }
    \caption{Beispiel einer S-Box im AES-Algorithmus}
\end{table}

\subsection{ShiftRows}
\textit{ShiftRows} verschiebt die Reihen der 4 x 4-Matrix zyklisch: Die $n$-te Reihe wird um $(n-1)$ Stellen nach links verschoben.
\[
\begin{array}{c}
\begin{bmatrix}
c_0  & c_1  & c_2  & c_3  \\
c_4  & c_5  & c_6  & c_7  \\
c_8  & c_9  & c_{10} & c_{11} \\
c_{12} & c_{13} & c_{14} & c_{15}
\end{bmatrix}
\quad\rightarrow\quad
\begin{bmatrix}
c_0  & c_1  & c_2  & c_3  \\
c_5  & c_6  & c_7  & c_4  \\
c_{10} & c_{11} & c_8  & c_9  \\
c_{15} & c_{12} & c_{13} & c_{14}
\end{bmatrix}\\
\text{Beispiel einer Shift Row Tabelle}
\end{array}
\]

\subsection{MixColumns}
MixColumns sorgt für eine Mischung der Spalten, indem jede Spalte der Matrix als Polynom im Galois-Feld \(GF(2^8)\) betrachtet wird. Jede Spalte wird mit einem festen Polynom \(a(x) = 03x^3 + 01x^2 + 01x + 02\) multipliziert, wobei die Multiplikation modulo \(x^4 + 1\) erfolgt. Dieser Schritt verteilt die Bits der Eingabe, um eine stärkere Diffusion zu gewährleisten.

\subsection{Add Round Key}
In Add Round Key wird jede Byte-Position des Blocks mit dem entsprechenden Byte des Rundenschlüssels XOR-verknüpft. Dieser Schritt sorgt dafür, dass der Hauptschlüssel in jede Runde einfließt.
