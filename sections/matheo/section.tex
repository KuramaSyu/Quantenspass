\section{Infos}
\setlength{\parindent}{0pt}

Advanced Encryption Standard (AES) ist ein Symmetrischer Blockchiffre Algorythmus zum 
Verschlüsseln von sensiblen daten. Z.B(VPN verbindungen, HTTPS anfragen sowie WLAN Verschlüsselungen(WPA2)).
Die Schlüssellängen beträgt entweder 128, 192 oder 256 Bit. Je länger der Schlüssel ist desto sicherher ist die Verschlüsselung.
Die zu verschlüsselnde Datengröße beträgt jeweils pro Block 128 Bit.
AES basiert auf mehreren Runden von Transformationen, die Daten verschlüsseln. Die Anzahl der Runden hängt von der Schlüssellänge ab.
\begin{itemize}
    \item 128-bit key \textendash\ 10 Runden
    \item 192-bit key \textendash\ 12 Runden
    \item 256-bit key \textendash\ 14 Runden
\end{itemize}

Es gibt jeweils einen anderen Schlüssel pro Runde, welcher von dem Hauptschlüssel berechnet wird.\\

Erfunden vom National Institute of Standards and Technology (NIST) im Jahr 2001.\\

AES ist ein Block Cipher.\\
Ein Block Cipher nimmt einen festen Bit-Wert ein und gibt einen festen Bit-Wert aus.\\

AES-Blöcke sind in einem 4 x 4 Byte (128 Bit) Block angelegt:\\

\[
\begin{bmatrix}
c_0  & c_4  & c_8  & c_{12} \\
c_1  & c_5  & c_9  & c_{13} \\
c_2  & c_6  & c_{10} & c_{14} \\
c_3  & c_7  & c_{11} & c_{15}
\end{bmatrix}
\]

Jede Verschlüsselungsrunde besteht aus 4 Schritten:\\
\begin{itemize}
    \item SubBytes
    \item ShiftRows
    \item MixColumns
    \item Add Round Key
\end{itemize}

Die letzte Runde hat keinen MixColumns-Schritt.\\

\subsection{SubBytes}

\subsection{Shift Rows}

\noindent
\flushleft
\(
\begin{bmatrix}
c_0  & c_1  & c_2  & c_3  \\
c_4  & c_5  & c_6  & c_7  \\
c_8  & c_9  & c_{10} & c_{11} \\
c_{12} & c_{13} & c_{14} & c_{15}
\end{bmatrix}
\quad \rightarrow \quad
\begin{bmatrix}
c_0  & c_1  & c_2  & c_3  \\
c_5  & c_6  & c_7  & c_4  \\
c_{10} & c_{11} & c_8  & c_9  \\
c_{15} & c_{12} & c_{13} & c_{14}
\end{bmatrix}
\)
\flushleft
\subsection{MixColumns}
\subsection{Add Round Key}
