\section{AES Algorythmus}
\setlength{\parindent}{0pt}


Advanced Encryption Standard (AES) ist ein Symmetrischer Blockchiffre Algorythmus zum 
Verschlüsseln von sensiblen daten. Z.B VPN verbindungen, HTTPS anfragen sowie WLAN Verschlüsselungen(WPA2).
Dieser algorythmus wurde von der National Institute of Standards and Technology (NIST) im Jahr 2001 entwikelt. 
Die Schlüssellängen beträgt entweder 128, 192 oder 256 Bit. Je länger der Schlüssel ist desto sicherher ist die Verschlüsselung.
Die zu verschlüsselnde Datengröße beträgt immer einen festen wert 128 Bit pro Block, da AES eine Blockchiffre ist und keine Stromchiffre,
 welche immer einen bit oder byte nimmt und diesen verschlüsselt 
AES basiert auf mehreren Runden von Transformationen, die Daten verschlüsseln. Die Anzahl der Runden hängt von der Schlüssellänge ab.
Pro Runde wird ein neuer Schlüssel verwendet, einen Rundenschlüssel, welcher anhand des Hauptschlüssels berechnet wird.
\begin{itemize}
    \item 128-bit key \textendash\ 10 Runden
    \item 192-bit key \textendash\ 12 Runden
    \item 256-bit key \textendash\ 14 Runden\\
\end{itemize}



AES-Blöcke sind in einem 4 x 4 Byte (128 Bit) Block angelegt:

\[
\begin{bmatrix}
c_0  & c_4  & c_8  & c_{12} \\
c_1  & c_5  & c_9  & c_{13} \\
c_2  & c_6  & c_{10} & c_{14} \\
c_3  & c_7  & c_{11} & c_{15}
\end{bmatrix}
\]

Jede Verschlüsselungsrunde besteht aus 4 Schritten:
\begin{itemize}
    \item SubBytes
    \item ShiftRows
    \item MixColumns
    \item Add Round Key
\end{itemize}

Sobald AES in der letzten runde der verschlüsselung ist wird der MixColumns schritt übersprungen, um die entschlüsselung zu ermöglichen.\\
\subsection{S-Box}
die S-Box (Substitution-Box) ist eine wesentliche komponente des AES und dient zur nichtlinearen Substitution von Bytes. 
Die S-Box ist da um die Entschlüsselung von unauthentivizierten benutzen zu erschwehren und die spuren vom schlüssel, dem 
input sowie dem verschlüsselten text zu verschleiern. Die S-Box wird vorher berechnet mit der Multiplikative Inverse im Galois-Feld GF(2⁸) 
wobei der exponent 8 die bit anzahl ist. jedes eingabe byte (außer 0, 0 ist ein vorbestimmter wert) wird inversiert, indem das byte in das Galois-Feld GF(2⁸) eingefügt wird 
Das bedeutet: $b' = b^{-1} \mod (x^8 + x^4 + x^3 + x + 1)$, wobei b das eingangs byte ist und b' das Multiplikatives Inverses von b. in AES wird standartmässig 
das polynom $x^8 + x^4 + x^3 + x + 1$ benutzt um die S-Box zu berechnen. Nach der Bestimmung des multiplikativen Inversen wird das 
Ergebnis einer festen linearen Transformation unterzogen, die zusätzlich eine Konstante addiert. Diese Transformation erfolgt bitweise und stellt sicher, dass die S-Box 
nichtlinear bleibt: $s_i = A * b_i +c$ wobei $s_i$ das final byte ist, A eine 8 x 8 Matrix, die eine feste lineare Abbildung darstellt,$b_i$ das eigabe 
byte und c ein constantes byte 



\subsection{SubBytes}
in SubBytes wird jeder wert eines bytes mit einem anderen byte ausgetauscht welche in der vorherigen sbox berechnet wurden. 
Die bytes werden in dem fall als addressen benutz und dann den wert der zugehörigen addresse zugeschrieben. Z.B \\
 \begin{table}[H]
    
    \resizebox{0.3\textwidth}{!}{
    \begin{tabular}{|c|c|c|c|c|c|c|c|}
    \hline
    63 & 7C & 77 & 7B & F2 & 6B & 6F & C5 \\
    \hline
    30 & 01 & 67 & 2B & FE & D7 & AB & 76 \\
    \hline
    CA & 82 & C9 & 7D & FA & 59 & 47 & F0 \\
    \hline
    AD & D4 & A2 & AF & 9C & A8 & 51 & A3 \\
    \hline
    40 & 8F & 92 & 9D & 38 & F5 & BC & B6 \\
    \hline
    DA & 21 & 10 & FF & F3 & D2 & CD & 0C \\
    \hline
    13 & EC & 5F & 97 & 44 & 17 & C4 & A5 \\
    \hline
    33 & 85 & 45 & F9 & 02 & 7F & 50 & 3C \\
    \hline
    \end{tabular}
    }
    \caption{Beispiel einer S-Box im AES-Algorithmus}
\end{table}


\subsection{Shift Rows}
In Shift Row werden die verschiedenen byte reihen verschoben. Der algorythmus verschiebt dabei jede reihe um $n-1$ stellen wobei n die reihe ist.

\noindent
\flushleft
\[
\begin{array}{c}
\begin{bmatrix}
c_0  & c_1  & c_2  & c_3  \\
c_4  & c_5  & c_6  & c_7  \\
c_8  & c_9  & c_{10} & c_{11} \\
c_{12} & c_{13} & c_{14} & c_{15}
\end{bmatrix}
\quad\rightarrow\quad
\begin{bmatrix}
c_0  & c_1  & c_2  & c_3  \\
c_5  & c_6  & c_7  & c_4  \\
c_{10} & c_{11} & c_8  & c_9  \\
c_{15} & c_{12} & c_{13} & c_{14}
\end{bmatrix}\\[1pt]
\\[1pt]
\text{Beispiel einer Shift Row Tabelle}
\end{array}
\]
\flushleft
\subsection{MixColumns}
\subsection{Add Round Key}
