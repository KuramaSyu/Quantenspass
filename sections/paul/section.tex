
Verschlüsselungsalgorithmen werden grundlegend in zwei Gruppen unterteilt.
Die erste Gruppe ist symmetrische Verschlüsselung, welche Grundlegend zum Verschlüsseln von Daten
oder Texten verwendet wird. Diese verwendet zum Ver\textendash\ und Entschlüssen den gleichen Schlüssel.
Im Gegensatz dazu steht die asymmetrische Verschlüsselung, welche zum Ver\textendash\ und Entschlüssen
zwei unterschiedliche Schlüssel verwendet. Diese werden Public Key (Verschlüsselung) und 
Private Key (Entschlüsselung) genannt. Als Beispiel, für einen Verwendungszweck von sowohl symmetrischer als auch 
asymmetrischer Verschlüsselung, ist folgend der HTTPS\textendash Verbindungsaufbau demonstriert.
\begin{figure}[h!]
    \centering
    \includegraphics[width=0.45\textwidth]{sections/paul/https_verbindungsaufbau.drawio.png}
    \caption{Demonstration des HTTPS Verbindungsaufbaus}
    \label{fig:http_verbindungsaufbau}
\end{figure}
Wie in Abbildung \ref{fig:http_verbindungsaufbau} dargestellt, erfolgt der Verbindungsaufbau, indem:
\begin{enumerate}
    \item Der Client unterstützte Verschlüsselungsalgorithmen an den Server sendet.
    \item Der Server das Sicherste der unterstützten Algorithmen auswählt und das Zertifikat, welches den Public Key enthält, an den Client sendet.
    \item Folgend generiert der Client einen symmetrischen Schlüssel, welcher mit dem Public Key des Servers verschlüsselt wird.
    \item Beide überprüfen die Integrität des Schlüssels gefolgt von der tatsächlichen Kommunikation.
\end{enumerate}

\subsection{Euklidischer Algorithmus: Beispiel zur Bestimmung des ggT}

Der Euklidische Algorithmus wird verwendet, um den größten gemeinsamen 
Teiler (ggT) zweier Zahlen effizient zu bestimmen. Anhand des Beispiels 
$N_1 = 132$ und $N_2 = 28$ wird der Algorithmus wie folgt angewendet:

\begin{enumerate}
    \item \textbf{Schritt 1:}  
    Teile die größere Zahl durch die kleinere und bestimme den Rest:  
    \[
    132 \div 28 = 4 \, \text{Rest} \, 20 \quad \Rightarrow \quad 132 = 4 \cdot 28 + 20.
    \]

    \item \textbf{Schritt 2:}  
    Die vorherige Divisorzahl $28$ wird der neue Dividend, und der Rest $20$ wird der neue Divisor:  
    \[
    28 \div 20 = 1 \, \text{Rest} \, 8 \quad \Rightarrow \quad 28 = 1 \cdot 20 + 8.
    \]

    \item \textbf{Schritt 3:}  
    Wiederhole den Vorgang, bis der Rest $0$ wird:  
    \[
    20 \div 8 = 2 \, \text{Rest} \, 4 \quad \Rightarrow \quad 20 = 2 \cdot 8 + 4,
    \]  
    \[
    8 \div 4 = 2 \, \text{Rest} \, 0 \quad \Rightarrow \quad 8 = 2 \cdot 4 + 0.
    \]
\end{enumerate}

\subsection{RSA Algorithmus}
\subsubsection{Grundlegende Funktionsweise von RSA}
RSA ist ein asymmetrischer Verschlüsselungsalgorithmus.
RSA wird benötigt, um eine gesicherte Kommunikation zwischen mindestens zwei Parteien aufzubauen, da 
es andernfalls keine Möglichkeit gäbe, den Schlüssel für die symmetrische Verschlüsselung \anf{abhörsicher} zu übertragen.
Die Sicherheit beruht darauf, dass Faktorisierung von Primzahlen, da dieses ein \anf{NP\textendash Hard} Problem ist\cite{moolchad_leveraging_nodate},
also nicht in polynomieller Zeit gelöst werden kann. 

\subsubsection{Verschlüsselung mit RSA}
Die zwei Primfaktoren, welche für RSA benötigt werden, 
werden folgend mit $p$ und $q$ bezeichnet, wobei $p$ und $q$ relativ Prim zueinander sind. 
$n$ ist das Produkt dieser beiden Primfaktoren, welches bei derzeitigen RSA\textendash\ Implementierungen 
2048 Bit lang ist. Im folgenden Beispiel wird zur Nachvollziehbarkeit für $p$ und $q$ jeweils 13 und 17 gewählt, womit
$n = p \cdot q = 221$ ist. Als Zweites wird die Eulersche Phi\textendash\ Funktion benötigt, um 
\[
\phi(n) = (p-1) \cdot (q-1) = 12 \cdot 16 = 192
\]
zu berechnen. Folgend wird $e$ als Teil des Public Keys gewählt, wobei $e$ und $\phi(n)$ teilerfremd sind. In diesem 
Beispiel wird $e = 23$ gewählt. Der Public Key ist damit bereits vollständig: $(e, n) = (23, 221)$. Die letzte benötigte
Variable ist $d$, welche der erste Teil des Private Keys ist. $d$ ist das multiplikative Inverse zu $e$ modulo $\phi(n)$,
also die Zahl für die gilt:
\[
d \cdot e \equiv 1 \pmod{\phi(n)}
\]
In diesem Fall ist $d = 167$, da $167 \cdot 23 = 3841 \equiv 1 \pmod{192}$. Der Private Key ist damit $(d, n) = (167, 221)$.
Damit das Wort \anf{Dresden} tatsächlich verschlüsselt werden kann, wird dieses zu ASCII umgewandelt, wie in 
Tabelle \ref{tab:encryption_dresden} in den ersten beiden Spalten dargestellt.
Die Verschlüsselung erfolgt für jedes Zeichen einzeln durch:
\[
c = m^e \bmod n
\]
wobei $m$ der ASCII Code des zu verschlüsselnden Zeichens ist und $c$ der verschlüsselte Wert.
Für das Wort \anf{Dresden} ergibt sich die Verschlüsselung zu $[204, 160, 186, 123, 42, 186, 219]$, wie in Tabelle
\ref{tab:encryption_dresden} in den Spalten \anf{$m^e \bmod n$}  und \anf{$c$} dargestellt.
\begin{table}[h]
    \centering
    \begin{tabular}{|c|c|c|c|}
        \hline
        Buchstabe & m (ASCII) & $m^e \bmod n$ & $c$ \\
        \hline
        D & 68 & $68^{23} \bmod 221$ & 204 \\
        r & 114 & $114^{23} \bmod 221$ & 160 \\
        e & 101 & $101^{23} \bmod 221$ & 186 \\
        s & 115 & $115^{23} \bmod 221$ & 123 \\
        d & 100 & $100^{23} \bmod 221$ & 42 \\
        e & 101 & $101^{23} \bmod 221$ & 186 \\
        n & 110 & $110^{23} \bmod 221$ & 219 \\
        \hline
    \end{tabular}
    \caption{Verschlüsselung der einzelnen Buchstaben}
    \label{tab:encryption_dresden}
\end{table}

\subsubsection{Entschlüsselung mit RSA}
Die Entschlüsselung erfolgt mit dem Private Key ($d, n$) analog durch:
\[
m = c^d \bmod n
\]
wie in Tabelle \ref{tab:decryption_dresden} dargestellt ist.
\begin{table}[H]
    \centering
    \begin{tabular}{|c|c|c|c|}
        \hline
        $c$ & $c^d \bmod n$ & m (ASCII) & Buchstabe \\
        \hline
        204 & $204^{167} \bmod 221$ & 68 & D \\
        160 & $160^{167} \bmod 221$ & 114 & r \\
        186 & $186^{167} \bmod 221$ & 101 & e \\
        123 & $123^{167} \bmod 221$ & 115 & s \\
        42 & $42^{167} \bmod 221$ & 100 & d \\
        186 & $186^{167} \bmod 221$ & 101 & e \\
        219 & $219^{167} \bmod 221$ & 110 & n \\
        \hline
    \end{tabular}
    \caption{Entschlüsselung der einzelnen Buchstaben}
    \label{tab:decryption_dresden}
\end{table}
Damit ist die Nachricht \anf{Dresden} erfolgreich verschlüsselt und wieder entschlüsselt worden.



% Python output
% p=13, q=17, n=221, phi=192, e=23, d=167
% Public Key: (23, 221)
% Private Key: (167, 221)
% Encrypting message: Dresden
% D[68] -> 68**23 % 221 = 204
% r[114] -> 114**23 % 221 = 160
% e[101] -> 101**23 % 221 = 186
% s[115] -> 115**23 % 221 = 123
% d[100] -> 100**23 % 221 = 42
% e[101] -> 101**23 % 221 = 186
% n[110] -> 110**23 % 221 = 219
% Decrypting message: [204, 160, 186, 123, 42, 186, 219]
% 204^167 % 221 = 68 -> D
% 160^167 % 221 = 114 -> r
% 186^167 % 221 = 101 -> e
% 123^167 % 221 = 115 -> s
% 42^167 % 221 = 100 -> d
% 186^167 % 221 = 101 -> e
% 219^167 % 221 = 110 -> n
% Original Message: Dresden
% Encrypted Message: [204, 160, 186, 123, 42, 186, 219]
% Decrypted Message: Dresden
