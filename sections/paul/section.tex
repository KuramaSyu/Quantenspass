\section{Kryptografie\textendash Grundlagen}
Verschlüsselungsalgorithmen werden grundlegend in zwei Gruppen unterteilt.
Die erste Gruppe ist symmetrische Verschlüsselung, welche Grundlegend zum Verschlüsseln von Daten
oder Texten verwendet wird. Diese verwendet zum ver\textendash und entschlüssen den gleichen Schlüssel.
Im Gegensatz dazu steht die asymmetrische Verschlüsselung, welche zum ver\textendash und entschlüssen
zwei unterschiedliche Schlüssel verwendet. Diese werden Public Key (Verschlüsselung) und 
Private Key (Entschlüsselung) genannt. Als Beispiel dazu folgend der HTTPS\textendash Verbindungsaufbau
\begin{enumerate}
    \item Der Handshakeprozess: Dieser Prozess erfolgt mit asymmetrischer Verschlüsselung. 
    Am Anfang des Verbindungsaufbau fordert der Client
    das Zertifikat des Servers an. Folgend prüft der Client das Zertifikat über eine Zertifizierungsstelle.
    Folgend wird für die weitere Kommunikation ein Schlüssel vom Client generiert. Dieser wird
    mit dem öffentlichen Schlüssel des Servers verschlüsselt. Der Server wird diesen folgend mithilfe des 
    eigenen Private Keys entschlüsseln und für die folgende symmetrische Verschlüsselung nutzen. Damit
    wurde auf sichere Weise ein "Passwort" übertragen.
    \item Kommunikation: In diesem Schritt erfolgt das Senden von tatsächlichen Daten. Zum Beispiel den
    HTML Code der Seite. Zum ver\textendash und entschlüsseln dieser Daten wird der Schlüssel verwendet,
    welcher in Schritt 1 generiert wurde.
\end{enumerate}
\section{Pauls Sektion - RSA}
RSA ist ein asymmetrischer Verschlüsselungsalgorithmus. Die Sicherheit von RSA beruht
auf der Faktorisierung von Primzahlen, da dieses ein NP-Hard Problem ist\cite{moolchad_leveraging_nodate}.


 