\section{Forschungsfrage}
Die Studie zielt darauf ab, die technischen und organisatorischen Auswirkungen 
von Quantencomputern auf die IT-Sicherheit zu untersuchen, mit besonderem Fokus 
auf kryptografische Verfahren. Dabei verfolgt sie einen Mixed-Methods-Ansatz, 
der sowohl technische Analysen als auch qualitative Befragungen kombiniert, um 
ein umfassendes Verständnis der Risiken und Handlungsoptionen zu gewinnen.

\subsection{Methodisches Vorgehen:}

Im Rahmen der technischen Analyse (FF1) werden Benchmarking-Tests für klassische 
Verschlüsselungsalgorithmen wie RSA und AES auf konventionellen Systemen durchgeführt. 
Diese Ergebnisse werden mit dokumentierten Benchmarks von Quantencomputern verglichen. 
Zusätzlich erfolgt eine mathematische Modellierung und Extrapolation der Zeitersparnisse 
bei verschiedenen Schlüssellängen, um kritische Schwellenwerte zu identifizieren. Die 
Performance-Unterschiede zwischen klassischer und quantenbasierter Kryptografie werden 
detailliert dokumentiert.

Die organisatorische Untersuchung (FF2) umfasst leitfadengestützte Experteninterviews 
mit IT-Sicherheitsverantwortlichen. Der Fokus liegt hierbei auf präventiven Maßnahmen 
und Anpassungsstrategien, die notwendig sind, um auf die Herausforderungen durch 
Quantencomputer zu reagieren. Die qualitativen Inhalte dieser Interviews werden analysiert, 
um Risiken und Handlungsoptionen aus der Sicht von Fachleuten zu bewerten.

\subsection{Erwartete Ergebnisse:}

Die Studie erwartet, die Performance-Unterschiede zwischen klassischer und Quantenkryptografie 
zu quantifizieren und kritische Schwellenwerte zu ermitteln, ab denen aktuelle Verschlüsselungen 
gefährdet sind. Zudem soll ein Katalog präventiver Maßnahmen entwickelt werden, basierend auf den 
Einschätzungen der Experten. Darüber hinaus wird eine zeitliche Dringlichkeit von Anpassungen 
eingeschätzt und Organisationen konkrete Handlungsempfehlungen zur Vorbereitung auf die Ära des 
Quantum Computing gegeben.

Der Mixed-Methods-Ansatz erlaubt es, die Risiken nicht nur technisch zu quantifizieren, sondern 
auch deren praktische Relevanz und organisatorische Bedeutung durch Expertenmeinungen einzuordnen. 
So soll die Studie einen umfassenden Beitrag zur Vorbereitung auf die Herausforderungen der 
Quantencomputing-Technologie leisten.