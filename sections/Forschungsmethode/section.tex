% \section{Forschungsfrage}
% Die Studie zielt darauf ab, die technischen und organisatorischen Auswirkungen 
% von Quantencomputern auf die IT-Sicherheit zu untersuchen, mit besonderem Fokus 
% auf kryptografische Verfahren. Dabei verfolgt sie einen Mixed-Methods-Ansatz, 
% der sowohl technische Analysen als auch qualitative Befragungen kombiniert, um 
% ein umfassendes Verständnis der Risiken und Handlungsoptionen zu gewinnen.

\section{Methodisches Vorgehen}

Die Untersuchung folgt einem Mixed-Methods-Ansatz, der sowohl technische als auch organisatorische Aspekte berücksichtigt. 
Im Rahmen der technischen Analyse (FF1) werden zwei klassische Verschlüsselungsalgorithmen, RSA und AES, auf 
konventionellen Systemen hinsichtlich ihrer Verschlüsselungs- und Entschlüsselungszeiten analysiert. Diese 
Tests berücksichtigen unterschiedliche Schlüssellängen und Datenmengen, um die Effizienz der Algorithmen unter 
variierenden Bedingungen zu bewerten, welche in Tabelle \ref{tab:schluessel_datenmengen} dargestellt sind. 
\begin{table}[ht]
    \centering
    \begin{tabularx}{0.45 \textwidth}{X X X}
    \hline
    \textbf{Algorithmus} & \textbf{Schlüssellängen} & \textbf{Datenmengen} \\
    \hline
    \text{RSA} & 1024, 2048, 4096, 8129 \text{ Bit} & 117, 245, 373, 501, 1013 \text{Byte}\\
    \hline
    \text{AES} & 128, 192, 256 \text{ Bit} & 1 \text{ KB}, 10 \text{ KB}, 100 \text{ KB}, 1 \text{ MB}, 5 \text{ MB}\\
    \hline
    \end{tabularx}
    \caption{Zu testende Schlüssellängen und Datenmengen}
    \label{tab:schluessel_datenmengen}
\end{table}
Die sehr spezifischen Datenmengen bei RSA beziehen sich auf die maximale Datenmenge pro Schlüssellaenge. 
Am Beispiel von 4096 Bit Verschlüsselung, würde diese zu Bytes umgerechnet werden. Diesen 
werden folgend 11 Bytes (konstant bei allen Schlüssellängen) abgezogen, da diese als sogenanntes Padding dienen. Damit ergibt sich $4096/8 - 11 = 501$. 
Bei AES beziehen sich die Datenmengen auf gängige zu Größen, welche bei HTTPS übertragen werden, da hierbei AES mit RSA kombiniert wird 
(Abbildung \ref{fig:http_verbindungsaufbau}).
Die Ergebnisse werden mit dokumentierten Benchmarks von Quantenalgorithmen 
wie Shor's und Grover's Algorithmus verglichen. Ergänzend werden analytische Methoden genutzt, um die theoretische 
Effizienz dieser Quantenalgorithmen zu modellieren. Diese Berechnungen basieren auf dokumentierten Benchmarks 
aktueller Quantencomputer und Zeitkomplexitätsmodellen, um kritische Schlüssellängen zu identifizieren, bei denen 
bestehende Verschlüsselungsverfahren potenziell gefährdet sind.

Die organisatorische Analyse (FF2) stützt sich auf qualitative Befragungen von IT-Sicherheitsverantwortlichen. 
In leitfadengestützten Interviews werden Einschätzungen zu den Risiken durch Quantencomputer 
sowie zur Dringlichkeit und Machbarkeit neuer kryptografischer Standards erhoben. Die Antworten werden thematisch 
analysiert, um zentrale Herausforderungen und Handlungsstrategien zu identifizieren. Ein besonderer Fokus liegt dabei 
auf präventiven Maßnahmen und ihrer zeitlichen Priorisierung. Beispielhafte Interviewfragen lauten:

\begin{itemize}
    \item „Wie schätzen Sie die Gefahr ein, die Quantencomputer für aktuelle Verschlüsselungsverfahren darstellen?“
    \item „Welche Maßnahmen halten Sie für besonders wichtig, um sich gegen diese Risiken abzusichern?“
    \item „Wie dringend ist Ihrer Meinung nach die Einführung von Post-Quanten-Kryptografie?“
    \item „Welche Herausforderungen sehen Sie bei der Umsetzung solcher Sicherheitsstandards?“
    \item „Gibt es Branchen, die Ihrer Meinung nach besonders von der Entwicklung der Quantencomputer betroffen sein werden?“
\end{itemize}

Die Ergebnisse beider Analyseschritte werden integriert, um ein umfassendes Verständnis der Risiken und 
Handlungsoptionen zu ermöglichen. Technische Erkenntnisse und Experteneinschätzungen werden kombiniert, um 
praxisorientierte Empfehlungen für Unternehmen zur Vorbereitung auf die Ära des Quantencomputings abzuleiten.\\

Das erwartete Ergebnis ist eine klare Einschätzung der Auswirkungen von Quantencomputern 
auf klassische Verschlüsselungsverfahren und die Identifikation von Risiken sowie 
Handlungsmöglichkeiten für Unternehmen. Durch das Benchmarking wird erwartet, dass RSA auf klassischen Computern nahezu 
unbrechbar ist, jedoch auf Quantencomputern exponentiell schneller gebrochen werden kann. 
Bei AES wird erwartet, dass der Algorithmus auf klassischen Computern 
eine hohe Sicherheit bietet, während Grover's Algorithmus die Effizienz 
der Entschlüsselung ungefähr halbiert, jedoch nicht exponentiell verringert.
Die Ergebnisse sollen praxisnahe Empfehlungen 
für den Übergang zu post-quanten-kryptografischen Standards liefern.
