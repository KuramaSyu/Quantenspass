\section{Fazit}
Die Analyse der Auswirkungen von Quantencomputern auf Verschlüsselungsalgorithmen
zeigt deutliche Bedrohungen auf die IT-Sicherheit. Insbesondere asymmetrische 
Verschlüsselungsverfahren wie RSA sind durch Shore's Algorithmus gefährdet. 
Der Grund liegt in der Fähigkeit von Quantencomputern, mit diesem Algorithmus
die Primfaktoren einer Zahl, wie zum Beispiel ein 2048 Bit RSA Schlüssel, 
effizient zu berechnen, wie in Sektion \ref{sec:shor} beschrieben wurde.\\
Auch AES ist durch den Grover-Algorithmus (siehe Sektion \ref{sec:grover}) bedroht, welcher die Rechenzeit 
halbiert, um den Schlüssel zu finden. Allerdings bietet dieser 
nicht wie Shore's Algorithmus eine exponentiell schnellere Berechnung und ist somit
grundlegend noch als sicher zu betrachten.\\
Bezüglich der ersten Forschungsfrage lässt sich festhalten, dass symmetrische 
Verschlüsselungsalgorithmen wie AES weniger stark betroffen sind als 
asymmetrische Verfahren. Während RSA durch Shor's Algorithmus praktisch 
gebrochen werden kann (siehe Sektion \ref{sec:shor}), bietet AES,
wie in Sektion \ref{sec:aes} dargestellt wurde, durch seine komplexe Struktur aus SubBytes, 
ShiftRows, MixColumns und der Rundenschlüssel auch gegenüber Quantencomputern 
noch Sicherheit, wenn auch mit erhöhten Anforderungen an die Schlüssellänge.\\
Die zweite Forschungsfrage hinsichtlich \anf{Store now, decrypt later-Angriffen}
zeigt die Dringlichkeit des Problems: Selbst wenn 
aktuelle Quantencomputer noch nicht leistungsfähig 
genug sind, sollten Unternehmen und Regierungen, bereits möglichst zeitnah 
Maßnahmen ergreifen, da verschlüsselte Daten gespeichert 
und später entschlüsselt werden könnten, wie in Sektion 
\ref{sec:quantencomputer} dargestellt wurde.\\

\subsection{Ausblick}
Die Forschung im Bereich der Quantenkryptographie stellt eine erhebliche 
Herausforderung für die IT-Sicherheit dar, da RSA praktisch gebrochen werden 
kann. Auch andere weit verbreitete Alternative Algorithmen wie \anf{Elliptic Curve \textendash ED25519},
sind durch den Shor's Algorithmus gefährdet. Dies betrifft indirekt auch symmetrische
Verschlüsselungsalgorithmen, da diese oft in Kombination 
mit asymmetrischen Verfahren, wie in Abbildung \ref{fig:http_verbindungsaufbau} dargestellt, 
verwendet werden. Das ist vor allem für die Kommunikationstechnik von Relevanz, da nun
Verbindungen kompromittiert werden können, da symmetrische Schlüssel mithilfe
von asymmetrischen Verfahren ausgetauscht werden.\\


