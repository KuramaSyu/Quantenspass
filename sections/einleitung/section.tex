\section{Einleitung}

Heutzutage werden Informationen von Firmen intern und international verschickt. 
Damit die Informationen sicher vor Dieben oder Konkurrenten bleiben, müssen sie verschlüsselt werden. 
Zur Verschlüsselung von Daten zwischen Computersystemen werden Methoden wie Rivest-Shamir-Adleman (RSA) oder der \textit{Advanced Encryption Standard} (AES) verwendet. 
Diese Verschlüsselungsmethoden basieren auf dem Konzept, dass es für konventionelle Computer zu lange dauern würde, die Berechnungen zur Entschlüsselung durchzuführen.

Da Quantencomputer in der Lage sind, deutlich schnellere Berechnungen durchzuführen, gibt es Probleme mit der Sicherheit der aktuellen Systeme. 
Das heißt, dass eine Partei, die Zugang zu einem Quantencomputer hat, der leistungsfähig genug ist, sich Zugang zu sensiblen Daten verschaffen könnte. Deshalb stellt sich die Frage:

\section*{Forschungsfrage 1}
Was sind die Auswirkungen von Quantencomputern auf aktuelle symmetrische und asymmetrische Verschlüsselungsalgorithmen, am Beispiel von RSA und AES?

\section*{Forschungsfrage 2}
Wie wird die Entwicklung von Quantencomputern die langfristige Sicherheit von Daten beeinflussen, und welche Strategien können Organisationen implementieren, um \textit{Store-now, decrypt-later}-Angriffe zu verhindern?
