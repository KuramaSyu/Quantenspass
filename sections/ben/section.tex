\section{Shor’s Algorithmus: Funktionsweise und mathematische Grundlage}

Shor’s Algorithmus ist ein Quantenalgorithmus, der entwickelt wurde, um 
effizient die Primfaktoren einer großen Zahl $N$ zu bestimmen. Dieses 
Problem bildet die Grundlage vieler kryptografischer Verfahren wie RSA, 
da es für klassische Computer äußerst schwierig ist, große Zahlen in 
ihre Primfaktoren zu zerlegen. Shor’s Algorithmus hingegen nutzt die 
Eigenschaften von Quantencomputern, um diese Aufgabe exponentiell 
schneller zu lösen. \cite{shor}

\subsection{Ablauf des Algorithmus}

\begin{enumerate}
    \item \textbf{Auswahl einer Basis $a$:}  
    Zunächst wird eine Zahl $a$ gewählt, wobei $1 < a < N$ gilt. Diese 
    Basis dient als Ausgangspunkt für die nachfolgenden Berechnungen.

    \item \textbf{Periodenfindung:}  
    Der Algorithmus sucht die kleinste positive Periode $r$, sodass die 
    Kongruenzgleichung  
    \[
    a^r \mod N = 1
    \]  
    erfüllt ist. Die Periode $r$ ist der Schlüssel zur Faktorisierung von $N$.  
    \begin{itemize}
        \item Es wird sichergestellt, dass $r > 0$ ist, da eine Periode von 
        $r = 0$ keine Aussagekraft hat.
        \item Zur Bestimmung von $r$ könnte man klassisch eine Wertetabelle 
        für $r$ und die entsprechenden Werte $z = a^r \mod N$ erstellen. Da 
        diese Methode jedoch ineffizient ist, wird auf einem Quantencomputer 
        die Quanten-Fourier-Transformation (QFT) verwendet, um $r$ effizient 
        zu berechnen.
    \end{itemize}

    \item \textbf{Verwendung der Periode $r$:}  
    Sobald $r$ bekannt ist, wird geprüft:  
    \begin{itemize}
        \item Falls $r$ ungerade ist, beginnt der Algorithmus erneut mit 
        einem anderen $a$.
        \item Ist $r$ gerade, werden die Primfaktoren von $N$ mit den 
        folgenden Formeln berechnet:  
        \[
        f_1 = \text{ggT}\left(a^{r/2} - 1, N\right) \quad \text{und} \quad f_2 = \text{ggT}\left(a^{r/2} + 1, N\right).
        \]  
        Der größte gemeinsame Teiler ($\text{ggT}$) wird dabei mithilfe 
        des Euklidischen Algorithmus ermittelt. \cite{shor_klassisch} \cite{shor_klassisch2} 
    \end{itemize}
\end{enumerate}

\subsection{Euklidischer Algorithmus: Beispiel zur Bestimmung des ggT}

Der Euklidische Algorithmus wird verwendet, um den größten gemeinsamen 
Teiler (ggT) zweier Zahlen effizient zu bestimmen. Anhand des Beispiels 
$N_1 = 132$ und $N_2 = 28$ wird der Algorithmus wie folgt angewendet:

\begin{enumerate}
    \item \textbf{Schritt 1:}  
    Teile die größere Zahl durch die kleinere und bestimme den Rest:  
    \[
    132 \div 28 = 4 \, \text{Rest} \, 20 \quad \Rightarrow \quad 132 = 4 \cdot 28 + 20.
    \]

    \item \textbf{Schritt 2:}  
    Die vorherige Divisorzahl $28$ wird der neue Dividend, und der Rest $20$ wird der neue Divisor:  
    \[
    28 \div 20 = 1 \, \text{Rest} \, 8 \quad \Rightarrow \quad 28 = 1 \cdot 20 + 8.
    \]

    \item \textbf{Schritt 3:}  
    Wiederhole den Vorgang, bis der Rest $0$ wird:  
    \[
    20 \div 8 = 2 \, \text{Rest} \, 4 \quad \Rightarrow \quad 20 = 2 \cdot 8 + 4,
    \]  
    \[
    8 \div 4 = 2 \, \text{Rest} \, 0 \quad \Rightarrow \quad 8 = 2 \cdot 4 + 0.
    \]
\end{enumerate}

Der letzte Divisor, bevor der Rest $0$ wird, ist der gesuchte größte gemeinsame Teiler:  
\[
\text{ggT}(132, 28) = 4. 
\] \cite{euklid}

\section{Felixs Sektion - Muss vergeben werden}
Keine Ahnung was ich schreiben soll\\
