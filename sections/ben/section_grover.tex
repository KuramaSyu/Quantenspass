\section{Der Grover-Algorithmus}
\label{sec:grover}
Der Grover-Algorithmus ist ein Quantenalgorithmus, der hilft, unstrukturierte Suchprobleme schneller zu lösen 
als klassische Methoden. Im Gegensatz zu klassischen Ansätzen, die $N$ Schritte erfordern, benötigt der 
Grover-Algorithmus nur etwa $\sqrt{N}$ Schritte, um die Lösung zu finden. Das macht ihn besonders relevant 
für die Sicherheit symmetrischer Verschlüsselungsverfahren wie AES.

\subsection{Grundprinzip}
Der Algorithmus arbeitet mit zwei Hauptschritten, die wiederholt angewendet werden:
\begin{itemize}
    \item Phasen-Inversion: Markiert die gesuchte Lösung durch Vorzeichenwechsel.
    \item Spieglung am Mittelwert: Verstärkt die Wahrscheinlichkeit, die gesuchte Lösung zu finden.
\end{itemize}

\subsection{Warum sind mehrere Iterationen nötig?}
Wenn die gesuchte Lösung markiert wird, wird diese zunächst nur mathematisch identifiziert, indem ihre Phase 
umgekehrt wird. Bei einer direkten Messung nach nur einem Durchlauf hätten jedoch alle Zustände noch ähnliche 
Wahrscheinlichkeiten, und die gesuchte Lösung könnte mit geringer Wahrscheinlichkeit übersehen werden. Die 
Spieglung am Mittelwert verstärkt deshalb gezielt die Amplitude der gesuchten Lösung, während die Amplituden der anderen 
Zustände verringert werden. 

Durch wiederholte Anwendung dieser beiden Schritte wird die Wahrscheinlichkeit, die richtige Lösung zu messen, 
Schritt für Schritt erhöht. Nach etwa $\sqrt{N}$ Iterationen ist diese Wahrscheinlichkeit so hoch, dass die Lösung 
mit nahezu 100\% Sicherheit gefunden wird.

\subsection{Funktionsweise}
\begin{enumerate}
    \item Initialisierung: Der Algorithmus beginnt mit einer Überlagerung aller möglichen Zustände:
        \[
        |\psi\rangle = \frac{1}{\sqrt{N}} \sum_{x=0}^{N-1} |x\rangle.
        \]

    \item Phasen-Inversion: Die gesuchte Lösung $x_{\text{target}}$ wird markiert, indem ihre Phase umkehrt wird:
        \[
        O |x\rangle = \begin{cases} 
            -|x\rangle & \text{wenn } x = x_{\text{target}}, \\
            |x\rangle & \text{sonst}.
        \end{cases}
        \]

    \item Spieglung am Mittelwert: Diese Operation verstärkt die Wahrscheinlichkeit der markierten Lösung:
        \[
        D |\psi\rangle = 2|\psi\rangle\langle\psi| - I.
        \]

    \item Wiederholung: Phasen-Inversion und Spieglung werden mehrfach angewendet (etwa $\sqrt{N}$-mal), um die 
    Wahrscheinlichkeit, die richtige Lösung zu messen, zu maximieren.
\end{enumerate}

\subsection{Beispiel: Suche in einer kleinen Liste}
Angenommen, wir suchen in einer Datenbank mit $N=4$ Einträgen nach der Lösung $x_{\text{target}}=2$. Die Schritte des 
Grover-Algorithmus über mehrere Iterationen lassen sich wie folgt erklären:

\begin{table}[H]
    \centering
    \begin{tabular}{|c|c|c|c|}
        \hline
        Iteration & Zustand & vor Diffusion & nach Diffusion \\
        \hline
        1 & $|0\rangle$ & $0.5$ & $0.25$ \\
        1 & $|1\rangle$ & $0.5$ & $0.25$ \\
        1 & $|2\rangle$ & $-0.5$ & $0.75$ \\
        1 & $|3\rangle$ & $0.5$ & $0.25$ \\
        \hline
        2 & $|0\rangle$ & $0.25$ & $0.125$ \\
        2 & $|1\rangle$ & $0.25$ & $0.125$ \\
        2 & $|2\rangle$ & $0.75$ & $0.875$ \\
        2 & $|3\rangle$ & $0.25$ & $0.125$ \\
        \hline
    \end{tabular}
    \caption{Amplitudenänderungen über zwei Iterationen im Grover-Algorithmus}
\end{table}

Nach der ersten Iteration wird die Amplitude der gesuchten Lösung ($|2\rangle$) von $-0.5$ auf $0.75$ verstärkt. 
Nach der zweiten Iteration steigt sie weiter auf $0.875$, wodurch die Wahrscheinlichkeit, diesen Zustand zu messen, deutlich zunimmt.
\\
Der Grover-Algorithmus reduziert die Sicherheit symmetrischer Verschlüsselungsverfahren wie AES. Während klassische 
Algorithmen $2^n$ Versuche benötigen, halbiert der Grover-Algorithmus diese Zahl auf $2^{n/2}$. Daher empfiehlt es sich, die 
Schlüssellängen zu verdoppeln, um quantensichere Standards zu gewährleisten \cite{grover}.

